\section{存在和唯一性定理}
\subsection{皮卡存在和唯一性定理}
\begin{defn}\label{def:Libsitz:like:}
设函数$f(x,y)$在区域$D$内满足不等式
$$\left|f(x,y_1)-f(x,y_2)\right|\leq L\left|y_1-y_2\right|,$$
其中常数$L>0$。则称函数$f(x,y)$在区域$D$内对$y$满足{\heiti 李卜西兹条件}(或简称{\heiti 李氏条件})。
\end{defn}\par
易知,若函数$f(x,y)$在凸形区域$D$内对$y$有连续的偏微商,并且$D$是有界闭区域,则$f(x,y)$在$D$内对$y$满足李氏条件;反之,结论不一定正确。例如,$f(x,y)=\left|y\right|$(对$y$)满足李氏条件,但当$y=0$时它对$y$没有微商。

\begin{theo}\label{theo:picard_theorem}
设初值问题
$$(E):\quad\quad \diff{y}{x}\,=\,f(x,y),\quad y(x_0)=y_0,$$
其中$f(x,y)$在矩形区域
$$R:\quad\quad |x-x_0|\leq a,\quad |y-y_0|\leq b$$
内连续,而且对$y$满足李氏条件。则$(E)$在区间$I=[x_0-h,x_0+h]$上有切只有一个解,其中常数
$$h=\min\left\{a,\frac{b}{M}\right\},\quad M>\max_{(x,y)\in R}|f(x,y)|.$$
\end{theo}
\begin{proof}
\begin{enumerate}[(1)]
\item 初值问题$(E)$等价于积分方程方程\eqref{proof:theo:picard_theorem:0}。\label{proof:theo:picard_theorem:1}
\item 用逐次迭代法构造皮卡序列。\label{proof:theo:picard_theorem:2}
\item 皮卡序列在区间$I$上一致收敛到积分方程的解。\label{proof:theo:picard_theorem:3}
\item 证明唯一性。\label{proof:theo:picard_theorem:4}
\end{enumerate}

步骤\eqref{proof:theo:picard_theorem:1}积分方程为
\begin{equation}\label{proof:theo:picard_theorem:0}
y\,=\,y_0+\int_{x_0}^xf(x,y)\D x.
\end{equation}\par
步骤\eqref{proof:theo:picard_theorem:2}使用逐次迭代法构造皮卡序列
\begin{equation}\label{proof:theo:picard_theorem:series}
y_{n+1}(x)\,=\,y_0+\int_{x_0}^xf(x,y_n(x))\D x\quad (x\in I)
\end{equation}
其中$n=0,1,2,\cdots$,$y_0(x)=y_0$。
由归纳法不难证明:由\eqref{proof:theo:picard_theorem:series}给出的皮卡序列$y=y_n(x)$在$I$上是连续的,而且满足不等式
$$|y_n(x)-y_0|\leq\left|\int_{x_0}^x|f(x,y_n(x))|\D x\right|\leq M|x-x_0|\quad (n=0,1,2,\cdots).$$\par
步骤\eqref{proof:theo:picard_theorem:3}皮卡序列$y_n(x)$的收敛性等价于级数
$$\sum_{n=1}^\infty[y_{n+1}(x)-y_n(x)]$$
的收敛性,利用李氏条件和归纳法假设可以得到\footnote{开学时自行补上步骤\eqref{proof:theo:picard_theorem:3}与步骤\eqref{proof:theo:picard_theorem:4}。}。
\end{proof}

\begin{theo}\label{theo:osgood}
设函数$f(x,y)$在区域$G$内连续按,而且满足不等式
$$|f(x,y_1)-f(x,y_2)|\leq F(|y_1-y_2|),$$
其中$F(r)>0$是$r>0$的连续函数,而且瑕积分
$$\int_0^{r_1}\frac{\D r}{F(r)}\,=\,\infty$$
($r_1>0$为常数)。则称$f(x,y)$在$G$内对$y$满足{\heiti Osgood条件}。\par
此时微分方程$\diff{y}{x}=f(x,y)$在$G$内经过每一点的解都是唯一的。
\end{theo}
\begin{proof}
假设不然。则在$G$内可以找到一点$(x_0,y_0)$使得微分方程有两个解$y=y_1(x)$和$$y=y_2(x)$$都经过$(x_0,y_0)$,而且至少存在一个值$x_1\neq x_0$,使得$y_1(x_1)\neq y_2(x_1)$。不妨设$x_1>x_0$,且$y_z(x_1)>y_2(x_1)$。令
$$\overline{x}\,=\,\sup_{x\in[x_0,x_1]}\{x:y_1(x)=y_2(x)\},$$
则显然有$x_0\leq\overline{x}<x_1$,而且
$$r(x)\overset{d}{=}y_1(x)-y_2(x)>0,\quad\overline{x}<x\leq x_1$$
和$r(\overline{x})=0$。因此,我们有
\begin{align}
r'(x) &= y_1'(x)-y_2'(x)=f(x,y_1(x))-f(x,y_2(x)) \\
&\leq F(|y_1(x)-y_2(x)|)=F(r(x)),
\end{align}
亦即
$$\frac{\D r(x)}{F(r(x))}\leq\D x\quad(\overline{x}<x\leq x_1).$$
从$\overline{x}$到$x_1$积分上式,得到
$$\int_0^{r_1}\frac{\D r(x)}{F(r(x))}\leq x_1-\overline{x},$$
其中$r_1=r(x_1)>0$。但这不等式的左端是$\infty$,而右端是一个有限的数。因此,这是一个矛盾,它证明了定理\ref{theo:osgood}。
\end{proof}

\subsection{佩亚诺存在定理}
\subsubsection{欧拉折线}
TODO
\subsubsection{Ascoli引理}
TODO
\subsubsection{培亚诺存在定理}
TODO

\subsection{解的延伸}
TODO

\subsection{比较定理及其应用}
TODO
