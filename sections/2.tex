\section{初等积分法}
\subsection{恰当方程}
考虑对称形式的一阶微分方程
\begin{equation}\label{eq:symform}
P(x,y)\D x+Q(x,y)\D y=0.
\end{equation}

如果存在一个可微函数$\varPhi(x,y)$,使得它的全微分为
\[
\D\varPhi(x,y)=P(x,y)\D x+Q(x,y)\D y,
\]
亦即它的偏导数为
\begin{equation}\label{eq:phipartialdiff}
\frac{\partial\varPhi}{\partial x}=P(x,y),\quad\frac{\partial\varPhi}{\partial y}=Q(x,y),
\end{equation}
则称\eqref{eq:symform}为\emphz{恰当方程}或\emphz{全微分方程}。因此,当方程\eqref{eq:symform}为恰当方程时,可将它改写为全微分的形式
\begin{equation*}
\D\varPhi(x,y)=P(x,y)\D x+Q(x,y)\D y=0.
\end{equation*}
从而
\begin{equation*}
\varPhi(x,y)=C
\end{equation*}
就是方程\ref{eq:symform}的一个通积分。
\begin{theo}\label{theo:approeq}% [微分方程是恰当方程的充要条件]
设函数$P(x,y)$和$Q(x,y)$在区域
\[
R:\quad\quad \alpha<x<\beta,\quad\gamma<y<\delta
\]

上连续,且有连续的一阶偏导数$\frac{\partial P}{\partial y}$与$\frac{\partial Q}{\partial x}$,则微分方程\eqref{eq:symform}是恰当方程的充要条件为恒等式
\begin{equation}\label{eq:approeq:ioi}
\frac{\partial P}{\partial y}(x,y)\equiv\frac{\partial Q}{\partial x}(x,y)
\end{equation}
在$R$内成立,并且当\eqref{eq:approeq:ioi}成立时,方程\eqref{eq:symform}的通积分为
\[
\int_{x_0}^xP(x,y)\D x+\int_{y_0}^yQ(x_0,y)\D y=C,
\]
或者
\[
\int_{x_0}^xP(x,y_0)\D x+\int_{y_0}^yQ(x,y)\D y=C,
\]
其中$(x_0,y_0)$是$R$中任意取定的一点。
\end{theo}

\begin{proof}
先证必要性。设方程\eqref{eq:symform}是恰当的,则存在函数$\varPhi(x,y)$,满足
\begin{equation}\label{proof:theo:approeq:1}
\frac{\partial\varPhi}{\partial x}=P(x,y),\quad\frac{\partial\varPhi}{\partial y}=Q(x,y).
\end{equation}
然后,我们在上面的第一式和第二式中,分别对$y$和$x$求偏导数,就可得到
\begin{equation}\label{proof:theo:approeq:2}
\frac{\partial P}{\partial y}=\frac{\partial^2\varPhi}{\partial y\partial x},\quad\frac{\partial Q}{\partial x}=\frac{\partial^2\varPhi}{\partial x\partial y}.
\end{equation}
由$\frac{\partial P}{\partial y}$和$\frac{\partial Q}{\partial x}$的连续性假设推知混合偏导数$\frac{\partial^2\varPhi}{\partial y\partial x}$和$\frac{\partial^2\varPhi}{\partial x\partial y}$是连续的,从而$\frac{\partial^2\varPhi}{\partial y\partial x}\equiv\frac{\partial^2\varPhi}{\partial x\partial y}$,因此由\eqref{proof:theo:approeq:2}式推得\eqref{eq:approeq:ioi}式。

再证充分性。设$P(x,y)$和$Q(x,y)$满足条件\eqref{eq:approeq:ioi},我们来构造可微函数$\varPhi(x,y)$,使\eqref{proof:theo:approeq:1}式成立。为了使\eqref{proof:theo:approeq:1}的第一式成立,我们可取
% 从这里定义\partialD,以前内容等有时间(Never)再修改。
\begin{equation}\label{proof:theo:approeq:3}
\varPhi(x,y)=\int_{x_0}^{x}P(x,y)\D x+\psi(y),
\end{equation}
其中函数$\psi(y)$待定,以使函数$\varPhi(x,y)$适合\eqref{proof:theo:approeq:1}的第二式。因此,由\eqref{proof:theo:approeq:3}得到
\[
\partialD{\varPhi}{y}=\partialD{}{y}\int_{x_0}^{x}P(x,y)\D x+\psi'(y)=\int_{x_0}^x\partialD{}{y}P(x,y)\D x+\psi'(y).
\]
再利用条件\eqref{eq:approeq:ioi}得到
\[
\partialD{\varPhi}{y}=\int_{x_0}^x\partialD{}{x}Q(x,y)\D x+\psi'(y)=Q(x,y)-Q(x_0,y)+\psi'(y).
\]
由此可见,为了使\eqref{proof:theo:approeq:1}的第二式成立,只要令$\psi'(y)=Q(x_0,y)$,亦即只要取\[
\psi(y)=\int_{y_0}^yQ(x_0,y)\D y
\]即可。这样,就得到了满足\eqref{proof:theo:approeq:1}的一个函数\[
\varPhi(x,y)=\int_{x_0}^xP(x,y)\D x+\int_{y_0}^yQ(x_0,y)\D y.
\]

如果在构造$\varPhi(x,y)$时,先考虑使\eqref{proof:theo:approeq:1}的第二式成立,则可用同样的方法,得到满足\eqref{proof:theo:approeq:1}的另一函数。因此,我们得到通积分如定理\ref{theo:approeq}。
\end{proof}

\subsection{变量分离的方程}
如果微分方程\eqref{eq:symform}中的函数$P(x,y)$和$Q(x,y)$均可分别表示为$x$的函数与$y$的函数的乘积,则称\eqref{eq:symform}为\emphz{变量分离的方程}。因此,只要令\[
P(x,y)=X(x)Y_1(y),\quad Q(x,y)=X_1(x)Y(y),
\]变量分离的方程可以写成如下的形式:
\begin{equation}\label{eq:varsep}
X(x)Y_1(y)\D x+X_1(x)Y(y)\D y=0.
\end{equation}

如果以因子$X_1(x)Y_1(y)$去除\eqref{eq:varsep}式的两侧,就得到\[
\frac{X(x)}{X_1}\D x+\frac{Y(y)}{Y_1(y)}\D y=0.
\]
此方程$x$与$y$互相分离,因此它的通积分为
\begin{equation}\label{eq:varsep:res}
\int\frac{X(x)}{X_1(x)}\D x+\int\frac{y(y)}{Y_1(y)}\D y=C.
\end{equation}
变量分离的方程\eqref{eq:varsep}的通积分由\eqref{eq:varsep:res}给出(要进行必要的不定积分运算);还要补上如下形式的特解(如果它们不在上述通积分之内的话):
\begin{equation*}
    \begin{cases}
        x=a_i\quad(i=1,2,\cdots) & \text{其中$a_i$是$X_1(x)=0$的根;} \\
        y=b_j\quad(j=1,2,\cdots) & \text{其中$b_j$是$Y_1(y)=0$的根。}
    \end{cases}
\end{equation*}

\subsection{一阶线性方程}
本节讨论\emphz{一阶线性}方程
\begin{equation}\label{eq:firstlinearode}
\frac{\D y}{\D x}+p(x)y=q(x).
\end{equation}
其中函数$p(x)$和$q(x)$在区间$I=(a,b)$上连续。当$q(x)\equiv0$时方程\eqref{eq:firstlinearode}成为
\begin{equation}\label{eq:firstlinearode:homo}
\frac{\D y}{\D x}+p(x)y=0.
\end{equation}
当$q(x)$不恒等于零时,称方程\eqref{eq:firstlinearode}为\emphz{非齐次}线性方程;而称方程\eqref{eq:firstlinearode:homo}为(相应的)\emphz{齐次}线性方程。

我们得到方程\eqref{eq:firstlinearode:homo}的解
\begin{equation}\label{eq:firstlinearode:homo:sol}
y=Ce^{-\int p(x)\D x}.
\end{equation}
现在要求解非齐次线性方程\eqref{eq:firstlinearode}。我们可把它改写为如下的对称形式:
\begin{equation}\label{eq:firstlinearode:sol}
\D y+p(x)y\D x=q(x)\D x.
\end{equation}
一般而言,\eqref{eq:firstlinearode:sol}不是恰当方程,但以因子$\mu(x)=e^{\int p(x)\D x}$乘\eqref{eq:firstlinearode:sol}两侧(注意$\mu(x)\neq 0$),得到方程
\[
e^{\int p(x)\D x}\D y+e^{\int p(x)\D x}p(x)y\D y=e^{\int p(x)\D x}q(x)\D x,
\]
它是全微分的形式
\[
\D(e^{\int p(x)\D x}y)=\D\int q(x)e^{\int p(x)\D x}\D x.
\]
由此可直接积分,得到通积分
\[
e^{\int p(x)\D x}y=\int q(x)e^{\int p(x)\D x}\D x+C.
\]
这样,就求出了方程\eqref{eq:firstlinearode:sol}的通解
\begin{equation}\label{eq:firstlinearode:sol2}
y=e^{-\int p(x)\D x}\Bigg(C+\int q(x)e^{\int p(x)\D x}\D x\Bigg),
\end{equation}

上述方法叫做\emphz{积分因子法},这是因为我们用因子$\mu(x)$乘微分方程\eqref{eq:firstlinearode:sol}的两侧后,它就转化为一个全微分方程,从而获得它的积分。
利用这种形式,容易得到初值问题
\begin{equation}\label{eq:firstlinearode:initval}
\frac{\D y}{\D x}+p(x)y=q(x),\quad y(x_0)=y_0
\end{equation}
的解为
\begin{equation}\label{eq:firstlinearode:initval:sol}
y=y_0e^{-\int_{x_0}^xp(t)\D t}+\int_{x_0}^xq(s)e^{-\int_s^xp(t)\D t}\D s,
\end{equation}
其中$p(x)$和$q(x)$在区间$I$上连续。

\subsection{初等变换法}
\subsubsection{齐次方程}
如果微分方程\eqref{eq:symform}中的函数$P(x,y)$和$Q(x,y)$都是$x$和$y$的同次(例如$m$次)齐次函数,即:
\begin{equation}\label{eq:firstlinearode:varsubs}
P(tx,ty)=t^mP(x,y),\quad Q(tx,ty)=t^mQ(x,y),
\end{equation}
则称方程\eqref{eq:symform}为\emphz{齐次方程}(注意这与上节定义的齐次线性方程不是一回事)。对于齐次方程\eqref{eq:symform},标准的变量替换是$y=ux$,其中$u$为新的未知函数,$x$认仍为自变量。从关系\eqref{eq:firstlinearode:varsubs}易知
\begin{equation*}
    \begin{cases}
        \quad P(x,y)=P(x,xu)=x^mP(1,u), &\\
        \quad Q(x,y)=Q(x,xu)=x^mQ(1,u), &
    \end{cases}
\end{equation*}
因此,把变换代入方程\eqref{eq:symform},就得
\[
x^m\left[P(1,u)+uQ(1,u)\right]\D x+x^{m+1}Q(1,u)\D u=0,
\]
这是一个变量分离的方程。

\emphz{【附注】\quad}易知方程\eqref{eq:symform}为齐次方程的一个等价定义是,它可以化为如下的形式:
\[
\frac{\D y}{\D x}=\varPhi(\frac yx).
\]

\begin{example}
讨论形如\[
\frac{\D y}{\D x}=f\left(\frac{ax+by+c}{mx+ny+l}\right)
\]的方程的求解法。这里设$a,b,c,m,n,l$为常数。
\end{example}

注意,当$c=l=0$时,它是齐次方程,因此可用变化$u=\sfrac{y}{x}$求解。当$c$和$l$不全为零时,可分如下两种情形讨论:

(1)\quad $\Delta=an-bm\neq 0.$

此时可选常数$\alpha$和$\beta$,使得
\[
\begin{cases}
\quad a\alpha+b\beta+c=0, &\\
\quad m\alpha+n\beta+l=0.
\end{cases}
\]
然后取自变量和未知函数的(平移)变换
\[
x=\xi+\alpha,\quad y=\eta+\beta,
\]
% 从这里定义\diff,以前内容等有时间(Never)再修改。
则原方程就化为$\xi$与$\eta$的方程
\[
\diff{\eta}{\xi}=f\left(\frac{a\xi+b\eta}{m\xi+n\eta}\right),
\]
这已是齐次方程。因此,只要令$u=\sfrac{\eta}{\xi}$,即可把它化成变量分离的方程。

(2)\quad $\Delta=an-bm=0.$

此时有$m/a=n/b=\lambda$。因此,原方程化为
\[
\diff{y}{x}=f\left(\frac{ax+by+c}{\lambda(ax+by)+l}\right).
\]
再令$v=ax+by$为新的未知函数,$x$仍为自变量,则上述方程可化为
\[
\diff{v}{x}=a+bf\left(\frac{v+c}{\lambda v+l}\right),
\]
它是一个变量分离的方程。

\subsubsection{伯努利方程}
\begin{defn}\label{def:eq:Bernoulli}
形如
\begin{equation}\label{eq:Bernoulli}
\diff{y}{x}+p(x)y=q(x)y^n
\end{equation}
的方程称为伯努利方程,其中$n$为常数,而且$n\neq 0$和$1$。
\end{defn}

以$(1-n)y^{-n}$乘方程\eqref{eq:Bernoulli}两边,即得
\[
(1-n)y^{-n}\diff{y}{x}+(1-n)y^{1-n}p(x)=(1-n)q(x).
\]
然后令$z=y^{1-n}$,就有
\[
\diff{z}{x}+(1-n)p(x)z=(1-n)q(x),
\]
这是关于未知函数$z$的一阶线性方程。

\subsubsection{里卡蒂方程}
\begin{defn}\label{def:eq:Riccati}
假如一阶微分方程\eqref{eq:firstode}的右端函数$f(x,y)$是一个关于$y$的二次多项式,则称此方程为二次方程;它可写成如下形式:
\begin{equation}\label{eq:Riccati}
\diff{y}{x}=p(x)y^2+q(x)y+r(x),
\end{equation}
其中函数$p(x),q(x),r(x)$在区间$I$上连续,而且$p(x)$不恒为零。方程\eqref{eq:Riccati}通畅又叫做里卡蒂(Riccati, 1676-1754)方程。
\end{defn}

\begin{theo}\label{theo:Riccati:sol}
设已知里卡蒂方程\eqref{eq:Riccati}的一个特解$y=\varphi_1(x)$,则可用积分法求得它的通解。
\end{theo}
\begin{proof}
对方程\eqref{eq:Riccati}作变换$y=u+\varphi_1(x)$,其中$u$是新的未知函数,代入方程\eqref{eq:Riccati},得到
\[
\diff{u}{x}+\diff{\varphi_1}{x}=p(x)\left[u^2+2\varphi_1(x)u+\varphi_1^2(x)\right]+q(x)\left[u+\varphi_1(x)\right]+r(x),
\]
由于$y=\varphi_1(x)$是\eqref{eq:Riccati}的解,从上式消去相关的项以后,就有
\[
\diff{u}{x}=\left[2p(x)\varphi_1(x)+q(x)\right]u+p(x)u^2,
\]
这是一个伯努利方程。因此,由前面对方程\eqref{eq:Bernoulli}的讨论可知,此方程可以用积分法求出通解。
\end{proof}

\subsection{积分因子法}
对于一般的方程\eqref{eq:symform},设法寻找一个可微的非零函数$\mu=\mu(x,y)$,舍得用它乘方程\eqref{eq:symform}后,所得方程
\begin{equation}\label{eq:intfactormethod:1}
\mu(x,y)P(x,y)\D x+\mu(x,y)Q(x,y)\D y=0
\end{equation}
成为恰当方程,亦即
\begin{equation}\label{eq:intfactormethod:2}
\partialD{(\mu P)}{y}=\partialD{(\mu Q)}{x}.
\end{equation}
这时,函数$\mu=\mu(x,y)$叫做方程\eqref{eq:symform}的一个\emphz{积分因子}。

事实上,寻求积分因子$\mu=\mu(x,y)$,就是求解偏微分方程\eqref{eq:intfactormethod:2},或等价地,求解一阶偏微分方程
\begin{equation}\label{eq:intfactormethod:3}
P\partialD{\mu}{y}-Q\partialD{\mu}{x}=\left(\partialD{Q}{x}-\partialD{P}{y}\right)\mu,
\end{equation}

对某些特殊情形,利用\eqref{eq:intfactormethod:3}去寻求\eqref{eq:symform}的积分因子是可行的。例如,假设方程\eqref{eq:symform}有一个只与$x$有关的积分因子$\mu=\mu(x)$,则由充要条件\eqref{eq:intfactormethod:3}推出
\[
Q\diff{\mu}{x}=\left(\partialD{P}{y}-\partialD{Q}{x}\right)\mu,
\]
或者
\begin{equation}\label{eq:intfactormethod:4}
\frac{1}{\mu(x)}\diff{\mu(x)}{x}=\frac{1}{Q(x,y)}\left(\partialD{P(x,y)}{y}-\partialD{Q(x,y)}{x}\right).
\end{equation}
由于上式左端只与$x$有关,所以右端亦然。因此,微分方程\eqref{eq:symform}有一个只依赖于$x$的积分因子的必要条件是:表达式
\begin{equation}\label{eq:intfactormethod:5}
\frac{1}{Q(x,y)}\left(\partialD{P(x,y)}{y}-\partialD{Q(x,y)}{x}\right).
\end{equation}
只依赖于$x$,而与$y$无关。

反之,设表达式\eqref{eq:intfactormethod:5}只依赖于$x$,记为$G(x)$。考虑到\eqref{eq:intfactormethod:4}式,我们令
\[
\frac{1}{\mu(x)}\diff{\mu(x)}{x}=G(x),
\]
由此得到
\begin{equation}\label{eq:intfactormethod:6}
\mu(x)=e^{\int G(x)\D x},
\end{equation}
容易验证它就是\eqref{eq:symform}的一个积分因子。
\begin{theo}\label{theo:intfactormethod:1}
微分方程\eqref{eq:symform}有一个只依赖于$x$的积分因子的充要条件是:表达式\eqref{eq:intfactormethod:5}只依赖于$x$,而与$y$无关;而且若把表达式\eqref{eq:intfactormethod:5}记为$G(x)$,则由\eqref{eq:intfactormethod:6}所示的函数$\mu(x)$是方程\eqref{eq:symform}的一个积分因子。
\end{theo}
\begin{theo}\label{theo:intfactormethod:2}
若$\mu=\mu(x,y)$是方程\eqref{eq:symform}的一个积分因子,使得
\[
\mu P(x,y)\D x+\mu Q(x,y)\D y=\D\varPhi(x,y),
\]
则$\mu(x,y)g(\varPhi(x,y))$也是\eqref{eq:symform}的一个积分因子,其中$g(\cdot)$是任意可微的(非零)函数。
\end{theo}

\subsection{应用举例}
\begin{example}
求已知曲线族的等角轨线。

假设在$(x,y)$平面上由方程
\begin{equation}\label{eq:isotrajectory:1}
\varPhi(x,y,C)=0
\end{equation}
给出一个以$C$为参数的曲线族。我们设法求出另一个曲线族
\begin{equation}\label{eq:isotrajectory:2}
\varPsi(x,y,K)=0,
\end{equation}
其中$K$为参数,使得族\eqref{eq:isotrajectory:2}中的任一条曲线与族\eqref{eq:isotrajectory:1}中的每一条曲线相交成定角$\alpha$($-\frac{\pi}{2}<\alpha\leq\frac{\pi}{2}$,以逆时针方向为正)。称这样的曲线族\eqref{eq:isotrajectory:2}为已知曲线族\eqref{eq:isotrajectory:1}的\emphz{等角轨线族}。特别,当$\alpha=\frac{\pi}{2}$时,称曲线族\eqref{eq:isotrajectory:2}为\eqref{eq:isotrajectory:1}的正交轨线族。
\end{example}

假设$\varPhi_C'\neq 0$,则可由联立方程
\begin{equation}\label{eq:isotrajectory:3}
\varPhi(x,y,C)=0,\quad \varPhi_x'(x,y,C)\D x+\varPhi_y'(x,y,C)\D y=0
\end{equation}
消去$C$,得到曲线族\eqref{eq:isotrajectory:1}所满足的微分方程
\begin{equation}\label{eq:isotrajectory:4}
\diff{y}{x}=H(x,y),
\end{equation}
其中
\[
H(x,y)=-\frac{\varPhi_x'(x,y,C(x,y))}{\varPhi_y'(x,y,C(x,y))},
\]
这里$C=C(x,y)$是由$\varPhi(x,y,C)=0$决定的函数。

如果我们把方程\eqref{eq:isotrajectory:4}在点$(x,y)$的线素斜率记为$y_1'$,而把与它相交成$\alpha$角的线素斜率记为$y'$。则当$\alpha\neq\frac{\pi}{2}$时,有
\[
\tan\alpha=\frac{y'-y_1'}{1+y'y_1'},
\]
即
\[
y_1'=\frac{y'-\tan\alpha}{y'\tan\alpha+1};
\]
因为$y_1'=H(x,y)$,所以等角轨线的微分方程为
\[
\frac{y'-\tan\alpha}{y'\tan\alpha+1}=H(x,y),
\]
亦即
\begin{equation}\label{eq:isotrajectory:5}
\diff{y}{x}=\frac{H(x,y)+\tan\alpha}{1-H(x,y)\tan\alpha}
\end{equation}

而当$\alpha=\frac{\pi}{2}$时,就有
\[
y'=-\frac{1}{y_1'},
\]
亦即所求正交轨线的微分方程为
\begin{equation}\label{eq:isotrajectory:6}
\diff{y}{x}=-\frac{1}{H(x,y)}.
\end{equation}
求解微分方程\eqref{eq:isotrajectory:5}(或\eqref{eq:isotrajectory:6}),就可以得到\eqref{eq:isotrajectory:1}的等角轨线族(或正交轨线族)\eqref{eq:isotrajectory:2}。